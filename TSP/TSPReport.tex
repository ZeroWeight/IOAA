\documentclass[UTF8,a4paper]{paper}
\usepackage{alltt}
\usepackage{amsmath}
\usepackage{bm}
\usepackage{color}
\usepackage{ctex}
\usepackage{float}
\usepackage{graphicx}
\usepackage{listings}
\usepackage{marvosym}
\usepackage{multicol}
\usepackage{xcolor}
\usepackage{url}
\title{\begin{large} 智能优化算法及其应用\end{large}
\\ TSP问题的优化}
\author{张蔚桐\ 2015011493\ 自55}
\begin{document}
\maketitle
\section{编码方式}
我们使用GA算法来优化传统的TSP问题,这是一个组合优化问题。
为了方便使用MATLAB自带的遗传算法工具箱,因此我们采用如下
方式来对组合进行编码。

我们假设所有的城市按照他们的编号构成一个循环队列,每次当
确定下一个访问的城市时,将其从循环队列中移除。我们引入
表征组合情况的向量$X$,其中$x(i)$表征从当前城市到下一个
城市之间在循环队列中的距离。举下面六个城市A\~F为例。

\{A,B,C,D,E,F\}此种访问顺序对应的向量即为[1,1,1,1,1,1]\\ 
而\{B,D,F,C,A,E\}这种访问顺序对应的向量为[2,2,2,2,2,2]
这样,就使得优化元素的向量的取值范围没有了限制,方便了GA
算法进行下一步的优化。GA算法我们采用了MATLAB的标准遗传算
法库,这样进一步减少了工作量。

\section{GA 优化效果}
如图\ref{F}是GA算法优化的结果,我们使用了GA算法优化了10个
城市,得到了很好的结果,在后来尝试对30,50,75个城市进行优化
的过程中,我们发现单凭GA算法很难取得非常令人满意的结论,因此
我们补充增加了另外一些优化方式,这一点将在后面说明。
\begin{figure}\centering
\includegraphics[width=\textwidth]{10.eps}
\caption{GA优化10城市TSP的最佳路线图和优化过程图}
\label{F}\end{figure}

\section{补充优化方法}
由于单凭GA算法我们很难获得尤其是75城市的最优优化效果,因此
我们采用了一些补充优化方式来进一步获得更好的结果。

首先是反复执行GA算法若干次,寻找GA算法能够提供的最优的路径
组合,通过这个方式,实验中我们发现环路的距离被明显的减少,
算法方差也由于反复执行的过程而降低,这种方式带来的代价是,
算法的执行时间被明显的增长了。

其次在得到GA能给出的最优路径之后,我们在这个路径的基础之上来
进一步的优化这条路径,首先我们进行交叉路径优化,也就是消除环路
中存在的交叉点。其次,我们遍历整个环路中的每一个城市,试图
交换每一个城市和他前一个城市的位置,如果路径变短则保留新的路径。
通过这两种方式,我们在很短的时间内将GA算法已经给出的路径明显
的变短了。经过简单的测试,我们得到的四种情况的最优路径为图
\ref{F1}-\ref{F4}所示。由于这种测试没有增加GA算法的迭代次数
(为节省时间),这些算法得到了路径可能稍劣于下一节算法测试中得到的
最优路径。
\section{算法稳定性}
我们花了大量时间重复执行了上述算法并尝试对于10城市,30城市,
50城市,75城市的TSP问题进行优化。每次优化过程我们先执行100次
GA算法,找出GA算法的最优解,然后反复执行交叉点删除和相邻城市
交叉方式对已有路径进行处理。通过这种方式得到了很好的10城市,30
城市,50城市,75城市的TSP问题解。我们对于每一个问题随机优化了
100次统计算法性能,算法性能如表\ref{T}所示。
四个问题的直方图如图\ref{F1hist} - \ref{F4hist}所示。

\begin{multicols}{2}
\begin{figure}[H]\includegraphics[width = \columnwidth]{L10.eps}
\caption{10城市TSP路径,环路距离2.69}\label{F1}\end{figure}
\begin{figure}[H]\includegraphics[width = \columnwidth]{L30.eps}
\caption{30城市TSP路径,环路距离461.78}\label{F2}\end{figure}
\begin{figure}[H]\includegraphics[width = \columnwidth]{L50.eps}
\caption{50城市TSP路径,环路距离537.83}\label{F3}\end{figure}
\begin{figure}[H]\includegraphics[width = \columnwidth]{L75.eps}
\caption{75城市TSP路径,环路距离656.427}\label{F4}\end{figure}
\end{multicols}


\section{算法特性分析}
首先,GA算法不需要考虑TSP问题的复杂性,在采用了上述相对合理的编码方式
之后,GA算法的交叉和变异均可以按照标准的GA算法流程进行设计。交叉实际上
变化为某一组城市之间的访问次序,而保证其他某些城市之间的访问次序不变。
因此,这种交叉有利于保留已有的优秀成分,有利于算法尽快收敛到最小值。

另外,GA算法可以实现并行化,在实际测试中采用了8个线程同时评估子代的性能。
相比于传统算法以及SA算法,实现并行化之后的GA算法相对较快。

然而,GA算法仍然很难直接实现对大规模TSP问题的直接求取,需要引入一些需要
数学推导的图论中的特性(包括欧几里得空间的假设等),这些都是需要改进的地方。

\begin{table}[h]\centering\caption{算法性能统计表}\label{T}
\begin{tabular}{|c|c|c|c|c|c|}\hline
城市数 & 最短优化距离 & 最长优化距离 & 距离中位数 & 距离平均值 & 距离方差\\ \hline 
10 & 2.6907 & 2.6907 & 2.6907 & 2.6907 & 0.0000 \\ \hline
30 & 444.9039 & 692.7088 & 546.0493 & 547.8904 & 46.7701 \\ \hline
50 & 500.3703 & 712.2952 & 586.7363 & 587.6895 & 41.8671 \\ \hline
75 & 670.2950 & 924.8517 & 781.7456 & 778.9307 & 51.9237 \\ \hline
\end{tabular}

\begin{tabular}{|c|c|c|c|}\hline
城市数 & 归一化标准差 &城市数 & 归一化标准差\\ \hline
10 & 0.0000 & 30 & 0.0854 \\ \hline
50 & 0.0712 & 75& 0.0667 \\ \hline
\end{tabular}\end{table}

\begin{multicols}{2}
\begin{figure}[H]\centering\includegraphics[width=\columnwidth]{hist10.eps}
\caption{10城市TSP优化长度直方图}\label{F1hist}\end{figure}
\begin{figure}[H]\centering\includegraphics[width=\columnwidth]{hist30.eps}
\caption{30城市TSP优化长度直方图}\label{F2hist}\end{figure}
\begin{figure}[H]\centering\includegraphics[width=\columnwidth]{hist50.eps}
\caption{50城市TSP优化长度直方图}\label{F3hist}\end{figure}
\begin{figure}[H]\centering\includegraphics[width=\columnwidth]{hist75.eps}
\caption{75城市TSP优化长度直方图}\label{F4hist}\end{figure}
\end{multicols}


\section{实验感想}
本次实验使用遗传算法GA实现组合优化TSP问题,相比于解决传统优化函数的问题,这种
问题涉及到合理的编解码等问题,使得问题的难度上升。同时,GA算法不依靠外部条件
的特性一方面保证了设计的简单性,一方面却使得可以通过进一步的专家经验的引入得到
更好的路径,因此我知道了在实际应用中,在使用尤其是像GA这种黑盒性很强的算法时,
需要适当的在优化过程中和优化后引入专家经验,这样使得我们可以获得更优解。

具体代码可以见\url{https://github.com/ZeroWeight/IOAA/TSP}
\end{document}