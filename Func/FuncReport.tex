\documentclass[UTF8,a4paper]{paper}
\usepackage{alltt}
\usepackage{amsmath}
\usepackage{bm}
\usepackage{color}
\usepackage{ctex}
\usepackage{float}
\usepackage{graphicx}
\usepackage{listings}
\usepackage{marvosym}
\usepackage{multicol}
\usepackage{xcolor}
\usepackage{url}
\title{\begin{large} 智能优化算法及其应用\end{large}
\\ 多极小函数的求取}
\author{张蔚桐\ 2015011493\ 自55}
\begin{document}
\maketitle
\section{引言}
我们采用遗传算法GA优化Griewank函数,Griewank函数的
的表达式如(\ref{eq1})式所示:
\begin{equation}\label{eq1}
f(\vec{\bm x}) = \sum_{i=1}^N\frac{x_i^2}{4000} 
- \prod_{i=1}^N\cos(\frac{x}{\sqrt i}) + 1
,\ \ x_i\in[-600,600]
\end{equation}
函数具有多个最小值,且函数可以拓展到高维空间上。
最小值取在$\vec{\bm x}=0$处。Griewank函数是典型的
非线性多模态函数,具有广泛的搜索空间,通常被认为是
优化算法很难处理的复杂多模态问题。

如图\ref{f1}-\ref{f3}是一维情况下Griewank函数图像
为了方便起见分别按照三个比例进行展示,从图\ref{f1}
中我们可以看出,函数在$[-600,600]$区间段明显可以看出
$x=0$为其最小值点,然而在图\ref{f2}中可以明显看出
函数在任意位置存在着明显的震荡情况,非常不利于传统算法
进行优化搜索。同时在图\ref{f3}中可以看到函数在
零点附近的很大范围内均仅剩几乎等幅的震荡过程,而函数
减除震荡之后所剩的基线基本都是平的。这给传统优化算法
提出了相当大的困难。

\begin{multicols}{3}
\begin{figure}[H]
\includegraphics[width=\columnwidth]{600.eps}
\caption{$x\in[-600,600]$时Griewank函数图像}
\label{f1}
\end{figure}
\begin{figure}[H]
\includegraphics[width=\columnwidth]{100.eps}
\caption{$x\in[-100,100]$时Griewank函数图像}
\label{f2}
\end{figure}
\begin{figure}[H]
\includegraphics[width=\columnwidth]{10.eps}
\caption{$x\in[-10,10]$时Griewank函数图像}
\label{f3}
\end{figure}
\end{multicols}
\section{遗传算法求取过程}
我们采用了遗传算法GA对函数的最小值点进行求取,其中
我们使用了MATLAB中自带的遗传算法函数库,这极大的方便了
我们的设计过程。由于Griewank函数适合向量化并行计算,
因此我们设计函数的评估体系为向量化评估,这从很大程度
提升了计算速度。其他设置我们采用了MATLAB的默认设置。

我们采取的优化维度为100维,这个维度下,函数的震荡等
问题非常明显,因此传统优化算法几乎不可能完成相关任务。
\section{随机试验统计结果}
我们对上述算法进行了100次随机试验,得到的优化最小值的
直方图图图\ref{hist}所示。优化最小值的平均值为0.0020,
标准差为0.0030,中位数为0.00084,极差为0.0135。
最优优化解达到0.00023,最差解达到0.0137。说明优化效果
还是很好的
\begin{figure}
\centering
\includegraphics[width=\textwidth]{hist.eps}
\caption{优化结果的直方图}
\label{hist}
\end{figure}
\section{某次试验优化目标变化曲线}
如图\ref{eg1},\ref{eg2}为某次试验优化目标的变化曲线,
这次实验中将目标优化到了0.00054这个位置。可以看到如图
\ref{eg1}所示,开始迭代几轮之后优化目标迅速降到1以下,
之后的下降逐渐开始趋于平缓,最优个体的下降速度开始放慢,
而平均个体震荡趋近最优解。
\begin{multicols}{2}
\begin{figure}[H]
\includegraphics[width=\columnwidth]{eg1.eps}
\caption{优化目标变化曲线}
\label{eg1}
\end{figure}
\begin{figure}[H]
\includegraphics[width=\columnwidth]{eg2.eps}
\caption{优化目标变化曲线}
\label{eg2}
\end{figure}
\end{multicols}
\section{遗传算法特点讨论}
遗传算法GA的特点使得他在求解这个问题时相对于其他的
智能优化算法和传统优化算法有着比较大的优势。首先
遗传算法受到邻域的影响较小,通过交叉变异等方式,
可以很快的跳出局部极小解。另外,从这个问题中我们可以
看出,各个维度之间的独立性相对较强,均是越趋近于0
越容易给出较优的解。因此通过将统一维度上的染色体交叉,
这些优势染色体会很好的保留下来,提升了搜索速度。

另外,遗传算法对优化函数的约束相对较小,同时可以做到
并行计算,这使得MATLAB将问题向量化提升计算速度变得
比较方便
\section{实验心得和体会}
这次实验通过使用GA算法对一个函数进行最小值优化,让我
初步了解了GA算法的操作流程和其特点。相比于之前了解的
一些经典优化算法,GA算法对优化目标的要求很低,甚至可
以将优化目标作为黑盒处理。然而,实际上通过加强对优化
目标的了解,进行对GA算法的设置,可以得到更好的计算
效果。当然,相对于传统优化算法,GA算法在一般情况下
计算时间也相对较长,这是由于GA算法没有获得更多的信息
所致的。

因此,传统优化算法在一些一般的优化问题中仍然有他的
实际意义与价值,在一些传统优化算法不能起到作用的地方,
GA等智能优化算法有着其不可或缺的重要作用。

具体代码实现可以见\url{https://github.com/ZeroWeight/IOAA/Func}
\end{document}